\documentclass[10pt]{beamer}

% Theme und Farben
\usetheme{Madrid}
\usecolortheme{default}

% Pakete
\usepackage[utf8]{inputenc}
\usepackage[T1]{fontenc}
\usepackage[ngerman]{babel}
\usepackage{amsmath}
\usepackage{amsfonts}
\usepackage{amssymb}
\usepackage{graphicx}
\usepackage{booktabs}
\usepackage{xcolor}

% Titel-Informationen
\title{Enzym-Kaskaden-Analyse}
\subtitle{Parameterbestimmung und Fehlerpfortpflanzung}
\author{Antonia Berger}
\date{17. November 2025}

% Logo (optional - auskommentiert, da Pfad möglicherweise nicht existiert)
% \logo{\includegraphics[height=1cm]{logo.png}}

\begin{document}

% Titelfolie
\begin{frame}
    \titlepage
\end{frame}
\section{Einleitung}
\begin{frame}{"Was bisher geschah"}
    \only<1>{\begin{figure}
            \centering
            \includegraphics[width=0.7\linewidth]{Figure/Aufbruch_Zusammenfassung(1).png}
        \end{figure}}
    \begin{itemize}
        \item Modellierung nicht linearer Austausch­raten zwischen flüssig-flüssig Phasen
        \item Rechnen von Austauschraten und Reactionen im equilibrierten Zustand
    \end{itemize}
\end{frame}
\begin{frame}{Überblick}
     \only<1>{\begin{figure}
            \centering
            \includegraphics[width=0.7\linewidth]{Figure/Aufbruch_Zusammenfassung(1).png}
        \end{figure}}
    \only<2>{\begin{figure}
            \centering
            \includegraphics[width=0.7\linewidth]{Figure/Aufbruch_Realität.png}
    \end{figure}}
    \textbf{Fokus Shift:}
    Vom Downstream Prozess zum Upstream Prozess
    \begin{itemize}
        \item Enzymkinetik
        \item Parameterbestimmung
        \item Digitaler Zwilling von Bioreaktoren
    \end{itemize}
\end{frame}
\section{Enzymkinetik}
\begin{frame}{Modellierung der Enzymkinetik}
    \begin{itemize}
        \item Multi Substrat Michaelis Menten Kinetik
        \item Jede Componente kann als Substrat und Inhibitor wirken
    \end{itemize}
    \begin{figure}
            \centering
            \includegraphics[width=0.7\linewidth]{Figure/mMM_mInh.png}
    \end{figure}

\end{frame}
\begin{frame}{Parameterbestimmung}
    \begin{figure}
            \centering
            \includegraphics[width=1\linewidth]{Figure/Reaktions_kaskade.png}
    \end{figure}
    \begin{itemize}
        \item Mit Julian zusammen
        \item Gegeben Daten von Enzymkinetik Experimenten
        \item Analyse von der Parameterbestimmung mit beruecksichtigung Messfehlern
    \end{itemize}
    \vfill
    \textbf{Frage:} Wie gut sind meine Parameterbestimmungen unter Berücksichtigung von Messfehlern?
\end{frame}
\begin{frame}{Parameterschätzung aus kinetischen Daten}
\begin{minipage}{0.45\textwidth}
    \begin{figure}
        \centering
        \includegraphics[width=1.0\linewidth]{figure/curve_fit.png}
        \label{fig:enter-label}
    \end{figure}
\end{minipage}
\hspace{1cm}
\begin{minipage}{0.45\textwidth}
\centering
     \begin{block}{Anpassungsparameter}
        Annahme:
       \begin{equation*}
           y_i - f(x_i,\theta) =   \varepsilon_i \text{ mit } \varepsilon_i \sim N(0,\sigma_i^2).
       \end{equation*}
       Finde $ \theta$ so dass 
       \begin{equation*}
           \min_{\theta} \sum_i ((f(x_i,\theta) - y_i)/ \sigma_i)^2
       \end{equation*}
   \end{block}
\end{minipage}
\begin{block}{Problem 1}
    Zwei Substrat Model mit Daten der Einzelsubstrat Kinetik zu fitten -> Potentiell schrierige Parameterschätzungen.
\end{block}
\begin{block}{Problem 2}
   Der fehler $y_i - f(x_i,\theta)$ ist nicht der direkte Messfehler, da $y_i$ in einem experimentellen Arbeitsablauf bestimmt wird.
\end{block}
\end{frame}
\begin{frame}{Berechnung von Reaktionsraten aus Plattenleserdaten}
    \only<1>{\begin{figure}
        \centering
        \includegraphics[width=0.7\linewidth]{figure/PR_workflow.jpeg}
        \label{fig:enter-label}
    \end{figure}}
    \only<2>{\begin{figure}
        \centering
        \includegraphics[width=0.7\linewidth]{figure/PR_workflow_with_error.jpeg}
        \label{fig:enter-label}
    \end{figure}}\pause
    \begin{block}{Fehlerquellen}
        \begin{itemize}
            \item Kalibrierungsfehler $\rightarrow$ Fehler bei der Berechnung von Reaktionsraten
            \item Pipettierfehler $\rightarrow$ Konzentrationsfehler
            \item Messfehler im Plattenleser $\rightarrow$ Fluoreszenzfehler
        \end{itemize}
    \end{block}
\end{frame}
\begin{frame}{Monte-Carlo-Bootstrap}
    \begin{block}{Ansatz}
    \begin{itemize}
        \item Implementierung eines Parameterschätzskripts, das verschiedene Arten von Fehlerquellen berücksichtigt
        \item Verwendung von Monte-Carlo-Bootstrap zur Schätzung von Parameterverteilungen
        \item Vergleich verschiedener Fehlermodelle und deren Einfluss auf Parameterverteilungen
    \end{itemize}
    \end{block}
\end{frame}
\begin{frame}{Ergebnisse Reaction 1 und 3}
    
\end{frame}
\begin{frame}{Ergebnisse Reaction 2}

\end{frame}
\section{Digitaler Zwilling}
\begin{frame}{PioReactor}
    \begin{itemize}
        \item Programmierbare Bioreaktor Plattform
        \item Ziel: Messen Modellieren Regeln mit MPC und MHE
        \item Just Starte: Mit Tobias Process aufsetzten
        \item Mit Jannis Messdaten in model integrieren
    \end{itemize}
\end{frame}


\end{document}