\documentclass[aspectratio=169]{beamer}
% Optional: Set a theme
\usetheme{Madrid}
\usecolortheme{seahorse}
\usepackage{graphicx} % Um Bilder einzufügen
\usepackage{subcaption}
\usepackage{tikz}
\usetikzlibrary{positioning}
% Paket für Zeichnungen
\usepackage{tikz}
\usetikzlibrary{arrows, positioning}
\usepackage{xcolor}
\usepackage{fontawesome}
\usepackage[version=4]{mhchem}
\usepackage{chemfig}
\usepackage{subcaption}
% Optional: Set some colors


\title{State Seminar 2025 II}
\author{Antonia Berger \& Jannis Bergmann}
\date{\today}

\begin{document}
\begin{frame}
    \maketitle
\end{frame}
\begin{frame}{Overview}
     \tableofcontents
\end{frame}
\begin{frame}{Previously on "State Seminar 2025" }
    \textbf{AUFBRUCH:}
        \only<1>{\begin{figure}
            \centering
            \includegraphics[width=0.7\linewidth]{stateSeminar2_25/figure/Aufbruch_Zusammenfassung.png}
            \label{fig:enter-label}
        \end{figure}}
        \only<2>{\begin{figure}
            \centering
            \includegraphics[width=0.7\linewidth]{figure/Aufbruch_Realität.png}
            \label{fig:enter-label}
        \end{figure}}\pause
    \textbf{Enzyme cascade:}
        \begin{figure}
            \centering
            \includegraphics[width=0.7\linewidth]{figure/Reaktions_kaskade.png}
            \label{fig:enter-label}
        \end{figure}
\end{frame}
\section{Parameter Estimation}
\begin{frame}{How good are my parameters?}
    \begin{block}{Given}
    \begin{itemize}
        \item Chemical drawing of enzyme cascade
        \item Model equations
        \item Estimated parameters and standard deviation
        \item Experimental data
    \end{itemize}
    \end{block}
    \begin{block}{Question}
        How "good" are these estimated parameters?
    \end{block}
\end{frame}
\begin{frame}{Estimate Parameters from Kinetic Data}
\begin{minipage}{0.45\textwidth}
    \begin{figure}
        \centering
        \includegraphics[width=1.0\linewidth]{figure/curve_fit.png}
        \label{fig:enter-label}
    \end{figure}
\end{minipage}
\hspace{1cm}
\begin{minipage}{0.45\textwidth}
\centering
  \begin{block}{Fitting parameters}
        Assumption:
       \begin{equation*}
           y_i = f(x_i,\theta)
       \end{equation*}
       Find $ \theta$ such that 
       \begin{equation*}
           \min_{\theta} \sum_i (f(x_i,\theta) - y_i)^2
       \end{equation*}
   \end{block}
\end{minipage}
\end{frame}
\begin{frame}{Michaelis-Menten Kinetics}
    \begin{block}{Enzyme-Substrate Reaction Mechanism}
    \begin{center}
        \ce{$S$ + $E$ <=>[$k_1$] $ES$ ->[$k_2$] $E$ + $P$}.
    \end{center}
    \end{block}
    \begin{minipage}{0.45\textwidth}
        \begin{block}{Assumptions}
            \begin{itemize}
                \item Steady-state approximation: $\frac{d[\text{ES}]}{dt} = 0$
            \end{itemize}
        \end{block}
        \hspace{5cm}
    \end{minipage}
    \hspace{1cm}
    \begin{minipage}{0.45\textwidth}
    \centering
        \begin{block}{Michaelis-Menten Equation}
            \begin{equation*}
                r = \frac{V_{\text{max}} \cdot [\text{S}]}{K_m + [\text{S}]}
            \end{equation*}
        \end{block}
        \begin{figure}
            \centering
        \includegraphics[width=0.7\linewidth]{stateSeminar2_25/figure/simple_michaelis_menten.png}
        \label{fig:enter-label}
        \end{figure}
    \end{minipage}
\end{frame}
\begin{frame}{Estimate Parameters from Kinetic Data}
\begin{minipage}{0.45\textwidth}
    \begin{figure}
        \centering
        \includegraphics[width=1.0\linewidth]{figure/curve_fit.png}
        \label{fig:enter-label}
    \end{figure}
\end{minipage}
\hspace{1cm}
\begin{minipage}{0.45\textwidth}
\centering
     \begin{block}{Fitting parameters}
        Assumption:
       \begin{equation*}
           y_i = f(x_i,\theta) + \varepsilon_i \text{ with } \varepsilon_i \sim N(0,\sigma_i^2).
       \end{equation*}
       Find $ \theta$ such that 
       \begin{equation*}
           \min_{\theta} \sum_i ((f(x_i,\theta) - y_i)/ \sigma_i)^2
       \end{equation*}
   \end{block}
\end{minipage}
\begin{block}{Problem}
    \begin{itemize}
        \item We can not assume that the errors $\varepsilon_i$ are normally distributed.
        \item $y_i - f(x_i,\theta)$ is not the direct measurement error since $y_i$ is determined in an experimental workflow.
    \end{itemize}
\end{block}
\end{frame}
\begin{frame}{Calculating reaction rates from plate reader data}
    \only<1>{\begin{figure}
        \centering
        \includegraphics[width=0.7\linewidth]{figure/PR_workflow.jpeg}
        \label{fig:enter-label}
    \end{figure}}
    \only<2>{\begin{figure}
        \centering
        \includegraphics[width=0.7\linewidth]{figure/PR_workflow_with_error.jpeg}
        \label{fig:enter-label}
    \end{figure}}\pause
\begin{block}{Approach}
    Implement parameter estimation script that accounts for different types of error sources
\end{block}
\end{frame}
\begin{frame}{Monte Carlo Bootstrap}
    \begin{block}{Source of error}
        \begin{itemize}
            \item Calibration errors $\rightarrow$ errors in calculating reaction rates
            \item Pipetting errors $\rightarrow$ concentration errors $c_i = c_{true} + \epsilon c_i$
            \item \textbf{Measurement errors} in plate reader $\rightarrow$ fluorescence errors $o_i = o_{true} + \epsilon o_i$
        \end{itemize}
    \end{block}
    \begin{block}{Monte Carlo Bootstrap Algorithm}
        \begin{enumerate}
            \item For $i = 1, \ldots, N$ bootstrap samples:
            \begin{enumerate}
                \item Sample errors: $\epsilon_{o_k} \sim N(0,\sigma_{o_k}^2)$
                \item Calculate perturbed values:  $o_k^{(i)} = o_k + \epsilon_{o_k}$
                \item Compute reaction rates with perturbed data
                \item Fit parameters $\theta^{(i)}$ to perturbed data
            \end{enumerate}
            \item Get parameter distribution from $\{\theta^{(1)}, \ldots, \theta^{(N)}\}$
            \item Calculate confidence intervals and parameter uncertainties, and correlations
        \end{enumerate}
    \end{block}
\end{frame}
\begin{frame}{Results for reaction 1 with noisy Plate Reader data}
\begin{minipage}{0.45\textwidth}
    \centering
    \begin{equation*}
    r_{1} = \frac{V_{\text{max,1}} \cdot [\text{NAD}] \cdot [\text{PD}]}{(K_{m,\text{PD}}+ [\text{PD}]) \cdot (K_{m,\text{NAD}} + [\text{NAD}])}
\end{equation*}
    {\small
    \begin{tabular}{|c|c|c|}
        \hline
        Parameter & Value ± $\sigma$  \\
        \hline
        $V_{\text{max,1}}$ & 0.0765 ± 0.0002\\
        $K_{m,\text{PD}}$ & 95.78 ± 0.24 \\
        $K_{m,\text{NAD}}$ & 1.927 ± 0.015 \\
        \hline
    \end{tabular}}
\end{minipage}
\hspace{0.5cm}
\begin{minipage}{0.45\textwidth}
\centering
   \begin{figure}
        \centering
        \includegraphics[width=0.9\linewidth]{figure/corner_plot_reaction_1_plate_reader.png}
        \label{fig:enter-label}
    \end{figure}
\end{minipage}
\end{frame}
\begin{frame}{Results for reaction 1 with noisy processed Data}

\begin{minipage}{0.45\textwidth}
\begin{equation*}
     r_{1} = \frac{V_{\text{max,1}} \cdot [\text{NAD}] \cdot [\text{PD}]}{(K_{m,\text{PD}}+ [\text{PD}]) \cdot (K_{m,\text{NAD}} + [\text{NAD}])}
\end{equation*}
\vspace{2cm}
    {\small
    \begin{tabular}{|c|c|c|c|}
        \hline
        Parameter & Value ± $\sigma$ \\
        \hline
        $V_{\text{max,1}}$ & 0.0764 ± 0.0002   \\
        $K_{m,\text{PD}}$ & 95.7672 ± 0.4926   \\
        $K_{m,\text{NAD}}$ & 1.9265 ± 0.0150  \\
        \hline
    \end{tabular}}
\end{minipage}
\hspace{0.5}
\begin{minipage}{0.45\textwidth}
\centering
   \begin{figure}
        \centering
        \includegraphics[width=0.9\linewidth]{figure/corner_plot_reaction_1_procced_data.png}
        \label{fig:enter-label}
    \end{figure}
\end{minipage}
\end{frame}
\begin{frame}{Results for reaction 2 with noisy Plate Reader data}
\begin{equation*}
    r_{2} = \frac{V_{\text{max,2}} \cdot [\text{Lactol}] \cdot [\text{NADH}]}{\left(K_{m,\text{Lactol}} \left(1 + \frac{[\text{PD}]}{K_{i,\text{PD}}}\right) + [\text{Lactol}]\right) \cdot (K_{m,\text{NADH}} + [\text{NADH}])}
\end{equation*}
\begin{minipage}{0.45\textwidth}
\centering
    {\small
    \begin{tabular}{|c|c|c|c|}
        \hline
        Parameter & Value ± $\sigma$  \\
        \hline
        $V_{\text{max,2}}$ & 11.57 ± 3.32  \\
        $K_{m,\text{Lactol}}$ & 105.6 ± 4.04  \\
        $K_{m,\text{NADH}}$ & 11.60 ± 3.50  \\
        $K_{i,\text{PD}}$ & 138.85 ± 4.06 \\
        \hline
    \end{tabular}}
\end{minipage}
\hspace{0.5cm}
\begin{minipage}{0.45\textwidth}
\centering
\begin{center}
   \begin{figure}
        \centering
        \includegraphics[width=0.9\linewidth]{figure/corner_plot_reaction2_plate_reader.png}
        \label{fig:enter-label}
    \end{figure}
\end{center}
\end{minipage}
\end{frame}
\begin{frame}{Results for reaction 2 with noisy processed Data}
\begin{equation*}
    r_{2} = \frac{V_{\text{max,2}} \cdot [\text{Lactol}] \cdot [\text{NADH}]}{\left(K_{m,\text{Lactol}} \left(1 + \frac{[\text{PD}]}{K_{i,\text{PD}}}\right) + [\text{Lactol}]\right) \cdot (K_{m,\text{NADH}} + [\text{NADH}])}
\end{equation*}
\begin{minipage}{0.45\textwidth}
\centering
        {\small\begin{tabular}{|c|c|c|c|}
        \hline
        Parameter & Value ± $\sigma$ \\
        \hline
        $V_{\text{max,2}}$ & 11.15 ± 0.88 \\
        $K_{m,\text{Lactol}}$ & 107.14 ± 0.87 \\
        $K_{m,\text{NADH}}$ & 11.10 ± 0.92 \\
        $K_{i,\text{PD}}$ & 140.45 ± 0.97  \\
        \hline
    \end{tabular}}
\end{minipage}
\hspace{0.5cm}
\begin{minipage}{0.45\textwidth}
\centering
   \begin{figure}
        \right
        \includegraphics[width=0.9\linewidth]{figure/corner_plot_reaction2_processed_data.png}
        \label{fig:enter-label}
    \end{figure}
\end{minipage}
\end{frame}
\begin{frame}{Results for reaction 3 with noisy Plate Reader data}
\begin{minipage}{0.45\textwidth}
\begin{equation*}
    r_{3} &= \frac{V_{\text{max,3}} \cdot [\text{Lactol}] \cdot [\text{NAD}]}{(K_{m,\text{Lactol}} + [\text{Lactol}]) \cdot (K_{m,\text{NAD}} + [\text{NAD}])}
\end{equation*}
\centering
   {\small \begin{tabular}{|c|c|c|c|}
        \hline
        Parameter & Value ± $\sigma^2$  \\
        \hline
        $V_{\text{max,3}}$ &  2.8755 ± 0.0059 \\
        $K_{m,\text{Lactol}}$ & 65.4094 ± 0.1227 \\
        $K_{m,\text{NAD}}$ & 3.0888 ± 0.0157 \\
        \hline
    \end{tabular}}
\end{minipage}
\hspace{0.5cm}
\begin{minipage}{0.45\textwidth}
\centering
   \begin{figure}
        \centering
        \includegraphics[width=0.9\linewidth]{figure/corner_plot_reaction3_plate_reader.png}
        \label{fig:enter-label}
    \end{figure}
\end{minipage}
\end{frame}
\begin{frame}{Results for reaction 3 with noisy processed data}
\begin{minipage}{0.45\textwidth}
\begin{equation*}
   r_{3} &= \frac{V_{\text{max,3}} \cdot [\text{Lactol}] \cdot [\text{NAD}]}{(K_{m,\text{Lactol}} + [\text{Lactol}]) \cdot (K_{m,\text{NAD}} + [\text{NAD}])}
\end{equation*}
        {\small\begin{tabular}{|c|c|c|c|}
        \hline
        Parameter & Value ± $\sigma$  \\
        \hline
        $V_{\text{max,3}}$ & 2.8753 ± 0.0095\\
        $K_{m,\text{Lactol}}$ & 65.4097 ± 0.29514 \\
        $K_{m,\text{NAD}}$ & 3.0885 ± 0.0240 \\
        \hline
    \end{tabular}}
\end{minipage}
\hspace{0.5cm}
\begin{minipage}{0.45\textwidth}
   \begin{figure}
        \centering
        \includegraphics[width=0.9\linewidth]{figure/corner_plot_reaction3_processed_data.png}
        \label{fig:enter-label}
    \end{figure}
\end{minipage}
\end{frame}
\begin{frame}{Summary and Outlook}    
    \begin{block}{Key Findings}
        \begin{itemize}
            \item Reaction 1: Very low uncertainties
            \item Reaction 2: Higher uncertainties, especially for $V_{\text{max,2}}$ and $K_{m,\text{NADH}}$ 
            \item Reaction 3: Mixed results - $K_{m,\text{NADH}}$ shows high uncertainty 
            \item Correlation results vary depending on the error source (plate reader vs processed data)
        \end{itemize}
    \end{block}
    
    \begin{block}{Outlook}
        \begin{itemize}
            \item Investigate \textbf{experimental design optimization} to reduce parameter correlations
            \item Optimize \textbf{uncertainty quantification pipeline}
            \item Simulate \textbf{cascade simulation with additional reactions} with CADET
            \item Keep learning 
        \end{itemize}
    \end{block}
\end{frame}
\section{PioReactor feat. Jannis Bergmann}
\begin{frame}{Part II}
   \tableofcontents
\end{frame}
\begin{frame}{Motivation}
    \begin{minipage}{0.45\textwidth}
    \begin{itemize}
        \item Extending CADET
        \begin{itemize}
            \item Simulates biotechnical processes
            \item Runs simulations "offline"
        \end{itemize}
        \item Development of a Digital Twin
    \end{itemize}
    \end{minipage}
    \begin{minipage}{0.45\textwidth}
        \begin{figure}
            \centering
            \includegraphics[width=0.75\linewidth]{stateSeminar2_25/figure/cadet_offline.png}
            \label{fig:enter-label}
        \end{figure}
    \end{minipage}

\end{frame}

\begin{frame}{CADET-Live*}
\begin{minipage}{0.45\textwidth}
    \begin{itemize}
        \item New module to extend CADET's features
        \item Running CADET during bio-processes
        \item Change process environment based on simulations
    \end{itemize}
\end{minipage}
    \begin{minipage}{0.45\textwidth}
        \begin{figure}
            \centering
            \includegraphics[width=0.75\linewidth]{stateSeminar2_25/figure/cadet_online.png}
            \label{fig:enter-label}
        \end{figure}
    \end{minipage}
\end{frame}

\begin{frame}{CADET-Live* (Features)}
    \begin{itemize}
        \item Written in C/C++
        \item Interfaces with CADET-Core
        \item Build Modular e.g. different methods of data input
        \item Efficient
        \item Open Source
    \end{itemize}
\end{frame}

\begin{frame}{Proof-of-concept with Pioreactor}
    \begin{column}{0.48\textwidth}
        \begin{itemize}
            \item Open Source mini bio-reactor
            \item Based on RaspberryPi
            \item Can be used standalone or in cluster
            \item Uses MQTT for communication
        \end{itemize}
    \end{column}
    \begin{column}{0.48\textwidth}
        \begin{figure}
            \centering
            \includegraphics[height=0.7\textheight]{stateSeminar2_25//figure/pioreactor-40ml.jpg}
            \caption{Pioreactor 40ml}
            \label{fig:pioreactor-40ml}
        \end{figure}
    \end{column}
\end{frame}

\begin{frame}{Current State}
    \begin{itemize}
        \item Pioreactor build
        \item Software setup completed
        \item Self-tests run successfully
        \item Getting familiar with software for CADET-Core and PioReactor
    \end{itemize}
\end{frame}

\begin{frame}{The End -  Thank you for your attention!}
\begin{block}{Summary}
    \begin{itemize}
        \item Parameter estimation and error propagation
        \begin{itemize}
            \item Example of estimating kinetic parameters from experimental data
            \item Where and how you assume measurement error matters.
        \end{itemize}
        \item PioReactor
        \begin{itemize}
            \item Small bio process which is easy to control
            \item Staring point to get familiar with feedback systems
        \end{itemize}
    \end{itemize}
\end{block}
\end{frame}

\end{document}