\documentclass[10pt]{beamer}

% Theme und Farben
\usetheme{Madrid}
\usecolortheme{default}

% Pakete
\usepackage[utf8]{inputenc}
\usepackage[T1]{fontenc}
\usepackage[ngerman]{babel}
\usepackage{amsmath}
\usepackage{amsfonts}
\usepackage{amssymb}
\usepackage{graphicx}
\usepackage{booktabs}
\usepackage{xcolor}

% Titel-Informationen
\title{Enzym-Kaskaden-Analyse}
\subtitle{Parameterbestimmung und Fehlerfortpflanzung}
\author{Antonia Berger}
\date{17. November 2025}

% Logo (optional - auskommentiert, da Pfad möglicherweise nicht existiert)
% \logo{\includegraphics[height=1cm]{logo.png}}

\begin{document}

% Titelfolie
\begin{frame}
    \titlepage
\end{frame}
\section{Einleitung}
\begin{frame}{"Was bisher geschah"}
    \only<1>{\begin{figure}
            \centering
            \includegraphics[width=0.7\linewidth]{Figure/Aufbruch_Zusammenfassung(1).png}
        \end{figure}}
    \begin{itemize}
        \item Modellierung nicht linearer Austauschraten zwischen flüssig-flüssig Phasen
        \item Rechnen von Austauschraten und Reaktionen im equilibrierten Zustand
    \end{itemize}
\end{frame}
\begin{frame}{Überblick}
     \only<1>{\begin{figure}
            \centering
            \includegraphics[width=0.7\linewidth]{Figure/Aufbruch_Zusammenfassung(1).png}
        \end{figure}}
    \only<2>{\begin{figure}
            \centering
            \includegraphics[width=0.7\linewidth]{Figure/Aufbruch_Realität.png}
    \end{figure}}
    \begin{itemize}
        \item Enzymkinetik Modellierung in CADET
        \item Parameterbestimmung unter Berücksichtigung von Messfehlern
        \item PioReactor
    \end{itemize}
\end{frame}
\section{Enzymkinetik}
\begin{frame}{Modellierung der Enzymkinetik}
    \only<1>{\begin{figure}
            \centering
            \includegraphics[width=1\linewidth]{Figure/Reaktions_kaskade.png}
    \end{figure}

       \small{ \begin{align*}
            r_1 &= \frac{V_{\max1} [PD] [NAD]}{(K_{m11} + [PD])(K_{m12} + [NAD])}\\
            r_2 &= \frac{V_{\max2} [NADH][HD]}{(K_{m21}(1+\frac{[PD]}{Ki}) + [HD]) (K_{m22} + [NADH])}\\
            r_3 &= \frac{V_{\max3} [HD] [NAD]}{(K_{m31} + [HD])(K_{m32} + [NAD])}\\
        \end{align*}}
    }
    \only<2-3>{
    \begin{minipage}{0.45\textwidth}
    \begin{itemize}
        \item Multi-Substrat Michaelis-Menten Kinetik
        \item Verschiedene Inhibitor Typen
        \item In CADET implementiert \\
        $ \to $ In verschiedenen Prozessen anwendbar \\
        $ \to $ Verwendbar mit anderen Reaktionsmodellen \\
        $ \to $ Sensitivitätsanalyse möglich
    \end{itemize}
    \end{minipage}
    \begin{minipage}{0.45\textwidth}
    \begin{figure}
            \centering
            \includegraphics[width=1.0\linewidth]{Figure/mMM_mInh.png}
    \end{figure}
    \end{minipage}

    \begin{equation*}
        \nu_{j} = v_{\mathrm{max},j} \prod_{i = 1}^{N_{sub,j}} \nu_{i,j} = v_{\mathrm{max},j} \prod_{i = 1}^{N_{sub,j}} \frac{ c_{i,j}}{K_{\mathrm{M}_{i,j}} + c_{i,j}}
    \end{equation*}}
    
    \only<3>{\begin{block}{Zum Beispiel: Kompetitiver Inhibitor}
        \begin{equation*}
        \only<3>{\nu_{i,j} =  \frac{c_{i,j}}{K_{\mathrm{M}_{i,j}}\,(1 + \sum_{k \in \mathcal{I}_{i,j}} \frac{c_{k}}{K^{c}_{I_{k}}}) + c_{i,j}}}
        \end{equation*}
    \end{block}}
    \only<4>{
        \begin{figure}
            \centering
            \includegraphics[width=0.7\linewidth]{Figure/EnzymKaskadeModelvsCadet.png}        
        \end{figure}
    }
\end{frame}
\begin{frame}{Parameterbestimmung}
    \begin{figure}
            \centering
            \includegraphics[width=1\linewidth]{Figure/Reaktions_kaskade.png}
    \end{figure}
    \begin{itemize}
        \item Mit Julian zusammen
        \item Gegeben Roh-Daten von Enzymkinetik Experimenten
        \item Analyse von der Parameterbestimmung mit Berücksichtigung von Messfehlern
    \end{itemize}
    \vfill
    \textbf{Frage:} Wie gut sind meine Parameterbestimmungen unter Berücksichtigung von Messfehlern?
\end{frame}
\begin{frame}{Parameterschätzung aus kinetischen Daten}
\begin{minipage}{0.4\textwidth}
    \begin{figure}
        \centering
        \includegraphics[width=1.0\linewidth]{figure/curve_fit.png}
        \label{fig:enter-label}
    \end{figure}
\end{minipage}
\hspace{0.5cm}
\begin{minipage}{0.50\textwidth}
\centering
     \begin{block}{Experimenteller Aufbau}
       \begin{equation*}
           \min_{\theta} \sum_i \left(\frac{f(c_i,\theta) - r_i}{ \sigma_i}\right)^2
       \end{equation*}
       \begin{itemize}
        \item $c_i$: Konzentrationsmessung zum Zeitpunkt $i$
        \item $f(c_i,\theta) = \frac{c_i \cdot V_{\max}}{c_i + K_m}$: Modellfunktion
        \item $\theta = (V_{\max}, K_m)$: Parameter
        \item $\sigma_i$: Standardabweichung der Messung
       \end{itemize}
   \end{block}
\end{minipage}
\begin{block}{Beobachtung}
    Durch den experimentellen Ablauf können verschiedene Arten von Messfehlern auftreten.
\end{block}
\end{frame}
\begin{frame}{Berechnung von Reaktionsraten}
    \only<1>{\begin{figure}
            \centering
            \includegraphics[width=0.7\linewidth]{Figure/detailierterWF1.png}
    \end{figure}}
    \only<2>{\begin{figure}
            \centering
            \includegraphics[width=0.7\linewidth]{Figure/detailierterWF2.png}
    \end{figure}}
    \begin{block}{Raten Berechnung}
        \begin{equation*}
            r = A [\text{U/mg}] = \frac{ m_{od} [\text{A}_{340}/\text{s}]\cdot 60 \text{ s/min} \cdot \phi_{well} \cdot \phi_{prod} }{ m_{cal} [ \text{A}_{340} \mu \text{M}] \cdot c_{prod}[\text{mg/L}]}
        \end{equation*}
    \end{block}
\end{frame}
\begin{frame}{Fehler Berücksichtigung}
    \begin{minipage}{0.6\textwidth}
    \begin{block}{Ansatz}
    \begin{itemize}
        \item Fehler kategorisiert
        \begin{itemize}
            \item Pipettierfehler $\rightarrow$ Konzentrationsfehler
            \item Messfehler im Plate Reader $\rightarrow$ Fluoreszenzfehler und Zeitfehler
        \end{itemize}
        \item Annahme: Fehler in einer Kategorie sind normalverteilt und unabhängig
        \item Parameterbestimmung aus Rohdaten wird der Workflow mit den jeweiligen Fehlern nachgebildet
        \item Monte Carlo Simulation zur Fehlerfortpflanzung
    \end{itemize}
    \end{block}
    \end{minipage}
    \begin{minipage}{0.35\textwidth}
    \begin{figure}
            \centering
            \includegraphics[width=0.7\linewidth]{Figure/Progamm_skizze.png}
    \end{figure}
    \end{minipage}
  
\end{frame}
\begin{frame}{Ergebnisse Reaktion 1}
    \begin{minipage}{0.45\textwidth}
        \begin{figure}
            \centering
            \includegraphics[width=1.0\linewidth]{Plots/reaction1_PD_full_vs_rate.png}
        \end{figure}
    \end{minipage}
    \begin{minipage}{0.45\textwidth}
        \begin{figure}
            \centering
            \includegraphics[width=1.0\linewidth]{Plots/reaction1_NAD_full_vs_rate.png}
        \end{figure}
    \end{minipage}
    \begin{minipage}{0.425\textwidth}
        \begin{block}{Annahmen}
            \begin{itemize}
                \item 1000 Monte Carlo Iterationen
                \item Pipettierfehler: 2 \%
                \item Plate Reader Fehler: 0.7 \%
            \end{itemize} 
        \end{block}
    \end{minipage}
    \hspace{0.5cm}
    \begin{minipage}{0.475\textwidth}
        \begin{block}{Ergebnisse}
            \centering
            \scriptsize{\begin{tabular}{|l|c|c|}
                        \hline
                        Param. & Wert (Vollständig) & Wert (Rate) \\
                        \hline
                        $V_{\max1}$ & 0.04 ± 0.0008 & 0.04 ± 0.009 \\
                        $K_{m11}$ & 77.88 ± 0.029 & 77.87 ± 0.16 \\
                        $K_{m12}$ & 70.24 ± 3.73& 80.62 ± 50.87  \\
                        \hline
                        \end{tabular}}
        \end{block}
    \end{minipage}
\end{frame}
\begin{frame}{Ergebnisse Reaktion 3}
    \begin{minipage}{0.45\textwidth}
        \begin{figure}
            \centering
            \includegraphics[width=1.0\linewidth]{Plots/reaction3_LACTOL_full_vs_rate.png}
        \end{figure}
    \end{minipage}
    \begin{minipage}{0.45\textwidth}
        \begin{figure}
            \centering
            \includegraphics[width=1.0\linewidth]{Plots/reaction3_NAD_full_vs_rate.png}
        \end{figure}
    \end{minipage}
    \begin{minipage}{0.45\textwidth}
        \begin{block}{Annahmen}
             \begin{itemize}
                \item 1000 Monte Carlo Iterationen
                \item Pipettierfehler: 2 \%
                \item Plate Reader Fehler: 0.8 \%
            \end{itemize} 
            
        \end{block}
    \end{minipage}
    \hspace{1cm}
    \begin{minipage}{0.45\textwidth}
       \begin{block}{Ergebnisse}
            \centering
            \scriptsize{\begin{tabular}{|l|c|c|}
                        \hline
                        Param. & Wert (Vollständig) & Wert (Rate) \\
                        \hline
                        $V_{\max3}$ & 1.734 ± 0.006 & 1.73 ± 0.006 \\
                        $K_{m31}$ & 61.93 ± 0.89 & 61.89 ± 0.931 \\
                        $K_{m32}$ & 3.38 ± 0.59 & 3.268 ± 0.08  \\
                        \hline
                        \end{tabular}}
        \end{block}
    \end{minipage}
\end{frame}
\begin{frame}{Ergebnisse Reaktion 2 I}
\begin{minipage}{0.45\textwidth}
        \begin{figure}
            \centering
            \includegraphics[width=1.0\linewidth]{Plots/reaction2_HP_full_vs_rate.png}
        \end{figure}
    \end{minipage}
    \begin{minipage}{0.45\textwidth}
        \begin{figure}
            \centering
            \includegraphics[width=1.0\linewidth]{Plots/reaction2_NADH_full_vs_rate.png}
        \end{figure}
    \end{minipage}
    \begin{minipage}{0.425\textwidth}
        \begin{block}{Annahmen}
            \begin{itemize}
                \item 1000 Monte Carlo Iterationen
                \item Pipettierfehler: 2 \%
                \item Plate Reader Fehler: 0.5 \%
            \end{itemize} 
        \end{block}
    \end{minipage}
    \hspace{0.5cm}
    \begin{minipage}{0.5\textwidth}
        \begin{block}{Ergebnisse}
            \centering
            \scriptsize{\begin{tabular}{|l|c|c|}
                        \hline
                        Parameter & Wert & Wert \\
                        \hline
                        $V_{max2}$ [U/mg] & 0.57 ± 0.03 &  0.57 ± 0.02 \\
                        $K_{m21}$ [mM] & 108.7 ± 18.4  & 107.8 ± 11.5 \\
                        $K_{m22}$ [mM] & 66.93 ± 91.73 & 2.14 ± 0.71 \\
                        $K_i$ [mM] & 119.82 ± 3.76 & 119.78 ± 4.74\\
                        \hline
                        \end{tabular}}
        \end{block}
\end{minipage}
\end{frame}
\begin{frame}{Ergebnisse Reaktion 2 II}
    \begin{minipage}{0.5\textwidth}
        \begin{figure}
            \centering
            \includegraphics[width=0.8\linewidth]{Plots/reaction2_PD_full_vs_rate.png}
        \end{figure}
    \end{minipage}
    \hspace{0.5cm}
    \begin{minipage}{0.4\textwidth}
        \begin{block}{Annahmen}
            \begin{itemize}
                \item 1000 Monte Carlo Iterationen
                \item Pipettierfehler: 2 \%
                \item Plate Reader Fehler: 0.5 \%
            \end{itemize}
        \end{block}
        
    \end{minipage}
\begin{block}{Ergebnisse}
            \centering
            \scriptsize{\begin{tabular}{|l|c|c|}
                        \hline
                        Parameter & Wert & Wert \\
                        \hline
                        $V_{max2}$ [U/mg] & 0.57 ± 0.03 &  0.57 ± 0.02 \\
                        $K_{m21}$ [mM] & 108.7 ± 18.4  & 107.8 ± 11.5 \\
                        $K_{m22}$ [mM] & 66.93 ± 91.73 & 2.14 ± 0.71 \\
                        $K_i$ [mM] & 119.82 ± 3.76 & 119.78 ± 4.74\\
                        \hline
                        \end{tabular}}
        \end{block}
\end{frame}
\begin{frame}{Nächste Schritte}
    \begin{itemize}
        \item Ergebnisse analysieren, diskutieren und adaptieren
        \begin{itemize}
            \item Sind die Fehlerannahmen realistisch? % Relatives oder absolutes Rauschen?
            \item Sind die Fehlergrößen realistisch?
            \item Parameterschätzungen mit Multi-Substrat Michaelis-Menten Kinetik
        \end{itemize}
        \item Enzyminaktivierung integrieren
    \end{itemize}
\end{frame}
\section{Digitaler Zwilling}
\begin{frame}{AUSBLICK: PioReactor}
    \textbf{Ziel}: Messen, Modellieren, Regeln in CADET \\
    mit Model Predictive Control und Moving Horizon Estimation
    \only<1>{\begin{figure}
            \centering
            \includegraphics[width=0.7\linewidth]{Figure/MPC_und_MPE.png}
            \caption{Aus Vorlesungsskript siehe [1]}
    \end{figure}}
     \only<2->{
    \begin{minipage}{0.5\textwidth}
        \vspace{1.5cm}
        \begin{itemize}
        \item Programmierbarer Bioreaktor als Prototyp
        \item Tobias Jülch (IB) Testprozess aufsetzen
        \item Jannis Bergmann (Student) Datenmanagement und CADET Schnittstelle entwickeln
    \end{itemize}
    \end{minipage}}
    \begin{minipage}{0.4\textwidth}
        \vspace{1.5cm}
        \only<2>{
        \begin{figure}
            \centering
            \includegraphics[width=1.0\linewidth]{Figure/PioReactor.jpg}
        \end{figure}}
        \only<3>{
        \begin{figure}
            \centering
            \includegraphics[width=1.0\linewidth]{Figure/CADET-LivePioreactor.drawio.png}
        \end{figure}}

    \end{minipage}
       
    
\end{frame}
\begin{frame}{Literatur}
    \begin{thebibliography}{1}
    \bibitem{ebenbauer2022}
    [1] C. Ebenbauer and M. Gharbi, \emph{Lectures on RECEDING HORIZON OPTIMAL CONTROL}, 2022.
    \end{thebibliography}
    
\end{frame}

\begin{frame}{Backup: CADET Simulation Vergleich}
    \begin{figure}
        \centering
        \includegraphics[width=1.0\linewidth]{Plots/cadet_simulation_full_vs_rate_noise.png}
    \end{figure}
\end{frame}
\begin{frame}
    \begin{minipage}{0.45\textwidth}
        \begin{figure}
            \centering
            \includegraphics[width=0.75\linewidth]{Plots/compare_ellipse_reaction1_NAD_full_vs_rate_noise_vmax_Km2.png}
        \end{figure}        
        \begin{figure}
            \centering
            \includegraphics[width=0.75\linewidth]{Plots/compare_ellipse_reaction1_PD_full_vs_rate_noise_vmax_Km2.png}
        \end{figure}

    \end{minipage}
    \begin{minipage}{0.45\textwidth}
        \begin{figure}
            \centering
            \includegraphics[width=0.75\linewidth]{Plots/compare_ellipse_reaction2_HP_full_vs_rate_noise_vmax_Km2.png}
        \end{figure}        
        \begin{figure}
            \centering
            \includegraphics[width=0.75\linewidth]{Plots/compare_ellipse_reaction2_NADH_full_vs_rate_noise_vmax_Km2.png}
        \end{figure}
    \end{minipage}
\end{frame}
\begin{frame}
    \begin{minipage}{0.45\textwidth}
        \begin{figure}
            \centering
            \includegraphics[width=0.75\linewidth]{Plots/compare_ellipse_reaction2_PD_full_vs_rate_noise_vmax_Km2.png}
        \end{figure}        
        \begin{figure}
            \centering
            \includegraphics[width=0.75\linewidth]{Plots/compare_ellipse_reaction2_PD_full_vs_rate_noise_vmax_Kmi.png}
        \end{figure}

    \end{minipage}
    \begin{minipage}{0.45\textwidth}
        \begin{figure}
            \centering
            \includegraphics[width=0.75\linewidth]{Plots/compare_ellipse_reaction3_LACTOL_full_vs_rate_noise_vmax_Km2.png}
        \end{figure}        
        \begin{figure}
            \centering
            \includegraphics[width=0.75\linewidth]{Plots/compare_ellipse_reaction3_NAD_full_vs_rate_noise_vmax_Km2.png}
        \end{figure}
    \end{minipage}
\end{frame}


\end{document}